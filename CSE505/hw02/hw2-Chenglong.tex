\documentclass[letter, notitlepage]{article}
\usepackage{bussproofs}
\usepackage{amssymb}
\usepackage{latexsym}
\usepackage{fancyhdr}
\usepackage{hyperref}
\usepackage{amsmath}
\usepackage[usenames,dvipsnames]{color} % Required for specifying custom colors and referring to colors by name
\usepackage{listings}
\usepackage[top=1in, bottom=1in, left=0.85in, right=0.85in]{geometry}
\usepackage{xcolor}
\usepackage{bera}
\usepackage{amsthm}
\usepackage{stackengine}
\usepackage[all,cmtip]{xy}

\usepackage[T1]{fontenc}
\usepackage[scaled]{beramono}

\newtheorem{theorem}{Theorem}
\newtheorem{lemma}{Lemma}
\newtheorem{definition}{Definition}
\newtheorem{property}{Property}

% Optional to turn on the short abbreviations
\EnableBpAbbreviations



\begin{document}
\title{CSE505 Homework\#2}
\date{}
\author{Chenglong Wang}
\maketitle

\section{Problem 1: Small-step Semantics}

In the following context, we use symbol $\mathit{H}$ for Heap, $e$ for expressions, $i$ for int values, $x$ for variables, and $l$ for lists of integers.


\paragraph{Heap} we define the heap $H$ as a map from variables to an pair of type ($\mathsf{Int}\times \mathsf{(list~Int)}$). i.e. the heap $H$ has the following type:
$$H:\mathsf{Var} \rightarrow \mathsf{Int} \times \mathsf{(list~Int)}$$

\paragraph{Eval} The function $\mathit{Eval}(H, e)$ is the evaluation function for an expression, as defined in the lecture, $Eval$ takes the current heap $H$ and an expression $e$, the result is either a boolean or an integer depending on the type of $e$. 

In the following rules, $H~|~s \to H'~|~s'$ can be pronounced as ``Evaluate one step on the statement $s$ under the current heap H will update the heap to $H'$ and the statement is reduced to $s'$''.

\begin{center}
\AXC{\vphantom{$|$}}\RightLabel{(Step-Nop)}
\UIC{$H~|~\mathsf{Seq}~\mathsf{Nop}~s \to H~|~s$}
\DP
~~~
\AXC{$\mathit{Eval}(H, e)=n$ ~~~~ $H~x = (i, l)$}\RightLabel{(Step-Assign)}
\UIC{$H~|~\mathsf{Assign}~x~e \to [x\mapsto(i, l)]H ~|~ \mathsf{Nop}$}
\DP
\end{center}

\begin{center}
\AXC{$H~|~s_1 \to H_1~|~s_1'$}\RightLabel{(Step-Seq)}
\UIC{$H~|~\mathsf{Seq}~s_1~s_2 \to H_1~|~\mathsf{Seq}~s_1'~s_2$}
\DP
~~~
\AXC{$\mathit{Eval}(H, e)=\mathtt{false}$}\RightLabel{(Step-Cond-F)}
\UIC{$H~|~\mathsf{Cond}~e~s \to H~|~\mathsf{Nop}$}
\DP
\end{center}

\begin{center}
\AXC{$\mathit{Eval}(H, e)=\mathtt{true}$}\RightLabel{(Step-Cond-T)}
\UIC{$H~|~\mathsf{Cond}~e~s \to H~|~s$}
\DP
~~~
\AXC{$\mathit{Eval}(H, e)=\mathtt{false}$}\RightLabel{(Step-While-F)}
\UIC{$H~|~\mathsf{While}~e~s \to H~|~\mathsf{Nop}$}
\DP
\end{center}

\begin{center}
\AXC{$\mathit{Eval}(H, e)=\mathtt{true}$} \RightLabel{(Step-While-T)}
\UIC{$H~|~\mathsf{While}~e~s \to H~|~\mathsf{Seq}~s~(\mathsf{While}~e~s)$}
\DP
\end{center}

\begin{center}
\AXC{$H~x = (i, l)$} \RightLabel{(Step-PushVar)}
\UIC{$H~|~\mathsf{PushVar}~x \to [x\mapsto(i, i::l)]H~|~\mathsf{Nop}$}
\DP
\end{center}

\begin{center}
\AXC{$H~x = (i, [~])$}\RightLabel{(Step-PopVar-E)}
\UIC{$H~|~\mathsf{PopVar}~x \to \mathsf{Nop}$}
\DP
~~~
\AXC{$H~x = (i, i'::l)$}\RightLabel{(Step-PopVar)}
\UIC{$H~|~\mathsf{PopVar}~x \to [x\mapsto(i',l)]H~|~\mathsf{Nop}$}
\DP
\end{center}

\section{Problem 2}
Before proving the theorem, we first define the function $\mathsf{nPushVar}:\mathsf{Var}\to \mathsf{Stmt} \to Int$.
\[
\begin{array}{l}
	\mathsf{nPushVar}~x~\mathsf{Nop}  := 0 \\
	\mathsf{nPushVar}~x~(\mathsf{Assign}~y~e) := 0\\
	\mathsf{nPushVar}~x~(\mathsf{Seq}~s_1~s_2)  := \mathsf{nPushVar}~x~s_1 + \mathsf{nPushVar}~x~s_2\\
	\mathsf{nPushVar}~x~(\mathsf{Cond}~e~s)  := \mathsf{nPushVar}~x~s\\
	\mathsf{nPushVar}~x~(\mathsf{While}~e~s)  := \mathsf{nPushVar}~x~s\\
	\mathsf{nPushVar}~x~(\mathsf{PushVar}~y)  := \mathtt{if}(x\equiv y)~\mathtt{then}~1~\mathtt{else}~0\\
	\mathsf{nPushVar}~x~(\mathsf{PopVar}~y) := 0
\end{array}
\]

\begin{lemma}\label{lemma:1} If a statement $s$ has no While loops, starting from the heap $H$ and  $H~|~s \to H'~|~s'$, then for all $x$, $\mathsf{length}~(\mathsf{snd}~(H'~x)) + \mathsf{nPushVar}~x~s' \leq \mathsf{length}~(\mathsf{snd}~(H~x)) + \mathsf{nPushVar}~x~s$. 
\end{lemma}
\begin{proof}
	Proof by induction on s.
	\begin{itemize}
		\item 
			\textbf{Case $s=\mathsf{Nop}$}: In this case, no reduction of $s$ can be derived, thus the property hold.
		\item 
			\textbf{Case $s=\mathsf{Seq}~s_1~s_2$}: In this case, as $H~|~s \to H'~|~s'$, by the rule Step-Seq, there exists $s_1'$ s.t. 
			\begin{equation}\label{eq:1}
				H~|~s_1 \to H'~|~s_1'
			\end{equation}
			and
			\begin{equation}\label{eq:2}
				s'=\mathsf{Seq}~s_1'~s_2
			\end{equation}
			Thus, by induction and (1), we have the following property:
			\begin{equation}\label{eq:3}
				\mathsf{length}~(\mathsf{snd}~(H'~x)) + \mathsf{nPushVar}~x~s'_1 \leq \mathsf{length}~(\mathsf{snd}~(H~x)) + \mathsf{nPushVar}~x~s_1
			\end{equation}
			According to the definition of $\mathsf{nPushVar}~x~s$, we have that 
			\begin{equation}\label{eq:4}
				\mathsf{nPushVar}~x~(\mathsf{Seq}~s_1~s_2) = \mathsf{nPushVar}~x~s_1 + \mathsf{nPushVar}~x~s_2
			\end{equation}
			and
			\begin{equation}\label{eq:5}
				\mathsf{nPushVar}~x~(\mathsf{Seq}~s'_1~s_2) = \mathsf{nPushVar}~x~s'_1 + \mathsf{nPushVar}~x~s_2
			\end{equation}
			By substitution $\mathsf{nPushVar}~x~s'_1$ and $\mathsf{nPushVar}~x~s_1$ using \ref{eq:4} and \ref{eq:5}, we have 
			\begin{equation}\label{eq:6}
				\mathsf{length}~(\mathsf{snd}~(H'~x)) + \mathsf{nPushVar}~x~(\mathsf{Seq}~s'_1~s_2) \leq \mathsf{length}~(\mathsf{snd}~(H~x)) + \mathsf{nPushVar}~x~(\mathsf{Seq}~s_1~s_2)
			\end{equation}
			Beside, as we have \ref{eq:2} and $s=\mathsf{Seq}~s_1~s_2$, replacing them in \ref{eq:6}, we have
			\begin{equation}\label{eq:7}
				\mathsf{length}~(\mathsf{snd}~(H'~x)) + \mathsf{nPushVar}~x~s' \leq \mathsf{length}~(\mathsf{snd}~(H~x)) + \mathsf{nPushVar}~x~s
			\end{equation}
			and then the case is proved.
		\item 
			\textbf{Case $s=\mathsf{Cond}~e~s_1$}.
			There are two subcases regarding to the value of $e$ evaluated under $H$.
			\begin{itemize}
				\item \textbf{Sub-Case $\mathsf{Eval}~H~e=\texttt{false}$}: According to the rule Step-Cond-F, after evaluation, $s'=Nop$ and $H'=H$. Thus 
					$$\mathsf{length}~(\mathsf{snd}~(H~x)) + \mathsf{nPushVar}~x~s' = \mathsf{length}~(\mathsf{snd}~(H~x))$$ and $$\mathsf{length}~(\mathsf{snd}~(H~x)) \leq \mathsf{length}~(\mathsf{snd}~(H'~x)) + \mathsf{nPushVar}~x~s$$
				So the property holds for this sub-case.
				\item \textbf{Sub-Case $\mathsf{Eval}~H~e=\texttt{true}$}
					According to the rule Step-Cond-T, $s'=s_1$ and $H'=H$. By definition of $\mathsf{nPushVar}$, we have $\mathsf{nPushVar}~x~(\mathsf{Cond}~e~s_1) = \mathsf{nPushVar}~x~s_1$.
					Thus we have 
					$$\mathsf{length}~(\mathsf{snd}~(H'~x)) + \mathsf{nPushVar}~x~s' = \mathsf{length}~(\mathsf{snd}~(H~x))$$ and $$\mathsf{length}~(\mathsf{snd}~(H~x)) \leq \mathsf{length}~(\mathsf{snd}~(H~x)) + \mathsf{nPushVar}~x~s$$
					And the sub-case is proved.
			\end{itemize}
			With both of the subcase proved, we have the case proved.
		\item \textbf{Case $s=\mathsf{While}~e~s_1$}. As we required that there is no $\mathsf{While}$ in $s$, this case does not exist.
		\item \textbf{Case $s=\mathsf{PushVar}~y$}.
			There are two subcases based on whether $y$ is $x$  or not.
			\begin{itemize}
				\item \textbf{Sub-case: $y\equiv x$}, in this case, $\mathsf{length} (H'~x) = \mathsf{length} (H~x) + 1$, while $\mathsf{nPushVar}~x~s = \mathsf{nPushVar}~x~s' + 1$.
					So we have 
					$$\mathsf{length}~(\mathsf{snd}~(H'~x)) + \mathsf{nPushVar}~x~s' = \mathsf{length}~(\mathsf{snd}~(H~x)) + \mathsf{nPushVar}~x~s$$
					and the sub-case is proved.
				\item \textbf{Sub-case: $y\neq x$}, in this case, after step-reduction, $\mathsf{snd}~(H~x)=\mathsf{snd}~(H'~x)$ and the number of $PushVar~x$ stay unchanged, so the lemma still holds.
			\end{itemize}
			Thus in the case the lemma holds.
		\item \textbf{Case $s=\mathsf{PopVar}~y$}. Similar to the previous case, there are two subcases regarding to the equivalence of $x$ and $y$. 
			\begin{itemize}
			\item $x\equiv y$, then the number of $\mathsf{nPushVar~x}$ remain the same but the length of $\mathsf{snd}~H~x$ is reduced by 1, which means that 
			$$\mathsf{length}~(\mathsf{snd}~(H'~x)) + \mathsf{nPushVar}~x~s' = \mathsf{length}~(\mathsf{snd}~(H~x)) - 1 + \mathsf{nPushVar}~x~s$$
			and the lemma holds.
			\item $x\neq y$, then both $\mathsf{snd}~(H~x)=\mathsf{snd}~(H'~x)$ and the number of $\mathsf{PushVar}~x$ stay unchanged, so the lemma holds.
			\end{itemize}
		Having these two sub-cases proved, the lemma holds in this case.
	\end{itemize}
	With all the case proved, the lemma hold for all $s$ and $H$.
\end{proof}

\begin{lemma}\label{lemma:2}
	If a statement $s$ with heap $H$ can step to $s'$ with heap status $H'$, then for all $x$, we have $\mathsf{length}~(\mathsf{snd}~(H'~x)) + \mathsf{nPushVar}~x~s' \leq \mathsf{length}~(\mathsf{snd}~(H~x)) + \mathsf{nPushVar}~x~s$.
\end{lemma}
\begin{proof}
	By induction on the number of derivation steps $n$.
	\begin{itemize}
	\item $n=1$, the case is proved according to the Lemma~\ref{lemma:1}.
	\item $n=k+1$, by induction, $s$ with $H$ can step $k$ steps to $s_1$ with $H_1$, and 
		\begin{equation}\label{eqn:9}
			\mathsf{length}~(\mathsf{snd}~(H_1~x)) + \mathsf{nPushVar}~x~s_1 \leq \mathsf{length}~(\mathsf{snd}~(H~x)) + \mathsf{nPushVar}~x~s.
		\end{equation}
		As we can step from $s_1$ with $H_1$ to $s'$ with $H'$ in one step, according to Lemma~\ref{lemma:1}, we have
		\begin{equation}\label{eqn:10}
			\mathsf{length}~(\mathsf{snd}~(H'~x)) + \mathsf{nPushVar}~x~s' \leq \mathsf{length}~(\mathsf{snd}~(H_1~x)) + \mathsf{nPushVar}~x~s_1.
		\end{equation}
		By combining \ref{eqn:9} with \ref{eqn:9}, we have 
		\begin{equation}
			\mathsf{length}~(\mathsf{snd}~(H'~x)) + \mathsf{nPushVar}~x~s' \leq \mathsf{length}~(\mathsf{snd}~(H~x)) + \mathsf{nPushVar}~x~s
		\end{equation}
		So that the case is proved.
	\end{itemize}
	Then we have the lemma proved.
\end{proof}

\begin{property}
If a statement $s$ has no While loops and from the empty heap $s$ can step to heap $h'$ and statement $s'$, then for all variables $x$, the length of the stack for $x$ in $h'$ does not exceed the number of $\mathsf{PushVar}~x$ statements in $s$ (the original statement).
\end{property}
\begin{proof}
 By applying Lemma~\ref{lemma:2} with $H=\{\forall~x.x\mapsto (0,[~])\}$, we have:
 	$$\mathsf{length}~(\mathsf{snd}~(~x)) + \mathsf{nPushVar}~x~s' \leq 0 + \mathsf{nPushVar}~x~s$$
 By reduction, we have 
 	$$\mathsf{length}~(\mathsf{snd}~(H'~x)) \leq \mathsf{nPushVar}~x~s - \mathsf{nPushVar}~x~s'$$
 as $\mathsf{nPushVar}~x~s'$ is alway greater than 0, we have:
 	$$\mathsf{length}~(\mathsf{snd}~(H'~x)) \leq \mathsf{nPushVar}~x~s$$
 So the property is proved.
\end{proof}

\end{document}